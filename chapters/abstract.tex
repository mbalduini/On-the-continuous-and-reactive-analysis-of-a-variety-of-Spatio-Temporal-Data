\chapter*{Abstract}

In recent years, an increasing number of situations call for reactive decision making process based on a heterogeneous streaming data.
In this context, the urban environment results particularly relevant thanks to the dense network of interactions between people and urban spaces that produces a great amount of spatio-temporal fast evolving data. Moreover, a modern city offers a multitude of stakeholders are interested in reactive decisions for urban planning, commuters, tourists.
The growing use of location-based social networks, and, in general, the diffusion of mobile devices improved the ability to create an accurate and up-to-date representation of reality (a.k.a. Digital footprint or Digital reflection or Digital twin) exploiting either social media or mobile phones data, i.e. Call Data Records (CDR).
However, better decisions can result from the analyses of multiple data sources simultaneously. 
In this context, we investigate the problem of how to create an holistic conceptual model to represent heterogeneous spatio-temporal data and how to develop a streaming computational model to enable reactive decisions.
The main outcomes of this research are \frappe{} conceptual model and \river{} streaming computational model with its implementations.

\frappe{} is a conceptual model, more precisely an ontology, that exploits digital image processing terms to model spatio-temporal data and to enable space, time, and content analysis.
It uses image processing common terms to bridge the gap between the data creators and visual interface creators and to enable visual analytics on spatio-temporal data.
We first formalize the spatial and temporal concepts in \frappe{} 1.0, and then we add concepts related to the provenance and the content in \frappe{} 2.0.
We check the adherence of both versions of \frappe{} to the five Tom Gruber's principles, and demonstrate the validity of the conceptual model in real world use cases.

\river{} is a streaming computational model that postpones the data transformation until a system can benefit from it (\textit{lazy transformation} approach). In this thesis we present \river{} from conceptual and logical point of view and propose an example of architecture for a system that implements it (physical plan).
We, then, present different implementations of \river{}, a single-threaded one (namely \sti{}) and two implementations based on distributed technologies (\sparkdi{} and \hivedi{}).
We demonstrate the validity of \river{} by evaluating its implementations against already available system and one against the other.

We propose real world use cases in Milan and Como to validate the approach of \frappe{} and \river{} in enabling visual analytics for spatio-temporal heterogeneous urban data.

Finally, we reflect on limitations and state the future directions of this research work.
In particular, those reflections involve the reasoning capabilities enabled by \frappe{}, the future evaluations of \river{} and the evolution of the graphic syntax we propose together with \river{}.



